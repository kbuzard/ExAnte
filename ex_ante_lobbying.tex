\documentclass[12pt]{article}

\addtolength{\textwidth}{1.4in}
\addtolength{\oddsidemargin}{-.7in} %left margin
\addtolength{\evensidemargin}{-.7in}
\setlength{\textheight}{8.5in}
\setlength{\topmargin}{0.0in}
\setlength{\headsep}{0.0in}
\setlength{\headheight}{0.0in}
\setlength{\footskip}{.5in}
\renewcommand{\baselinestretch}{1.0}
\setlength{\parindent}{0pt}
\linespread{1.1}

\usepackage{amssymb, amsmath, amsthm, bm}
\usepackage{graphicx,csquotes,verbatim}
\usepackage[backend=biber,block=space,style=authoryear]{biblatex}
\setlength{\bibitemsep}{\baselineskip}
\usepackage[american]{babel}
%dell laptop
\addbibresource{C:/Users/Kristy/Dropbox/Research/xBibs/tradeagreements.bib}
%\addbibresource{C:/Users/Kristy/Documents/Dropbox/Research/xBibs/tradeagreements.bib}
\renewcommand{\newunitpunct}{,}
\renewbibmacro{in:}{}


\DeclareMathOperator*{\argmax}{arg\,max}
\usepackage{xcolor}
\hbadness=10000

\newcommand{\ve}{\varepsilon}
\newcommand{\ov}{\overline}
\newcommand{\un}{\underline}
\newcommand{\ta}{\theta}
\newcommand{\al}{\alpha}
\newcommand{\Ta}{\Theta}
\newcommand{\expect}{\mathbb{E}}
\newcommand{\Bt}{B(\bm{\tau^a})}
\newcommand{\bta}{\bm{\tau^a}}
\newcommand{\btn}{\bm{\tau^n}}
\newcommand{\ga}{\gamma}
\newcommand{\Ga}{\Gamma}
\newcommand{\de}{\delta}

\begin{document}
\begin{center}
Ex-ante Lobbying
\end{center}
%\chead{Klimenko, Ramey and Watson (2007) summary.tex}

\vskip.2in
Ex-ante lobbying is lobbying before the trade agreement is signed to affect the level at which TA / applied tariff will be set
\begin{itemize}
	\item ex-post lobbying is lobbying after the TA is signed to change the applied tariff
	\item lobby has to pay both times
\end{itemize}

\vskip.2in
This set up says that governments set TA so that TA will be at optimal level at the time you're going to have to implement it
\begin{itemize}
	\item Ex-ante lobbying distorts you away from that being the optimal level for the government
	\item Instead makes it jointly optimal for the government and lobby (in MRC framework)
	\item Government's preferences are changed
	\item What welfare does each party get at TA stage?
		\begin{itemize}
			\item Government might get/lose political capital from different interested parties; otherwise, it's all forward looking---what will future gains and losses be
			\item Lobby just knows that this will bind its future action
		\end{itemize}
\end{itemize}

\vskip.5in
Adding ex-ante lobbying in case of endogenous lobbying ONLY
\begin{itemize}
	\item Without ex-ante lobbying, government chooses $\tau^A$ that leads to ex-post $e$ that maximizes ex-ante joint welfare
		\[
		  W(\ga(e),\tau^A) + W^*(\ga^*(e),\tau^A)
		\]
		knowing they'll set $\tau^N$ in the third stage (i.e. $\tau^N(\ga(e)) = \frac{8\ga(e) - 5}{68 - 8 \ga(e)} = \tau^R_{W,e}$).
		\begin{itemize}
			\item Set $\tau^A$ that will lead lobby to choose $e^A \Rightarrow \ga(e^A) \Rightarrow \tau^A$. That is, $\tau^A$ is the optimal unilateral choice of the home government when the lobby is faced with the constraint of the trade agreement.
			\item Lobby would max $\pi(\tau(\ga(e))) -e$; instead sets $e = e_{W,e} \Rightarrow \tau^R_{W,e}$ (i.e. $\tau^N(\ga(e_{W,e})) =\tau_{W,e}$)
		\end{itemize}
	\item Given this relationship, pick $\tau^A$ that maximizes joint welfare: call it $\tau^R_{W,e}$.
		\begin{itemize}
			\item i.e. set $\tau^{TA}$ optimally given what you expect will happen in the future: but commitment device allows you to control what happens in the future
		\end{itemize}
	\item Another way to say it: there's a welfare level for every $\ga$. Make the optimal $\ga$ happen / pick the $\ga$ you'd like to have
		\begin{itemize}
			\item You'll actually end up with a lower $e$ mapping into a lower $\ga$ to produce the associated $\tau$ according to $\tau^N$
		\end{itemize}
	\item With ex-post lobbying only, at ex-ante stage gov't has to take into account how ex-post lobbying will affect it; picks $\tau^{TA}$ to max joint political welfare subject to this constraint
\end{itemize}

\vskip.2in
Now, add in the ex-ante lobbying \`{a} la MRC.
\begin{itemize}
	\item Maximize joint welfare of both governments and both lobbies (but can concentrate on just one tariff because of separability)
		\[
		  max_{\tau,e} W(\ga(e),\tau) + W^*(\tau) + \pi(\tau) + \pi^*(\tau^*) - e - e^*
		\]
		\[
		  max_{\tau,e} CS_x(\tau) + (2+\ga(e)) \pi_x(\tau) + TR(\tau) + CS_x^*(\tau) + \pi_x^*(\tau) - e
		\]
		FOCs: 
		\begin{itemize}
			\item For $\tau$, same as in no ex-ante lobbying case, except now $2 + \ga(e)$ weight on profits.
			\item For $e$: $\frac{\partial \ga}{\partial e} \pi(\tau) = 1$
				\begin{itemize}
					\item In ex-post lobbying, $e$ will be determined by $\frac{\partial \pi}{\partial \tau}\frac{\partial \tau}{\partial \ga}\frac{\partial \ga}{\partial e} = 1$
					\item Do I need second term in ex-ante version?
				\end{itemize}
		\end{itemize}
	\item In MRC, there are essentially two $\ga$'s: one for ex-ante, one for ex-post
		\begin{itemize}
			\item But that doesn't match up with BS2005 framework, where the point of the TA is to take account of future $\ga$
			\item Also leads to bizarre outcome where government will take into account lobby's profits even if $e^{\text{EA}}=0$.
			\item They can max joint welfare because they have transferrable utility
			\item Something gets lost in MRC formulation about division of rents as a result of transferrable utility and this joint welfare max: they have tariff caps and strong bindings equivalent, even though these split the surplus in very different ways. There's no way to take account of this at the ex-ante stage.
		\end{itemize}
	\item To take ex-ante out of MRC, just take $x \cdot p$ out of maximand at ex-ante stage (they do it on page 1391)
		\begin{itemize}
			\item They don't treat case of immobile capital in 1998 paper, so can't compare there. But they show that there would be free trade in their model with no ex-ante bargaining (because of bargaining power assumption--lobbies have it all)
			\item I'm still not crystal clear on what purpose ex-ante lobbying serves in a world of fixed capital where gov't political welfare increases everywhere in $\ga$---here, gov't gives lobby whatever it wants ex-post
				\begin{itemize}
					\item Could $e^{EA}$ increase $\tau^{TA}$ in this world?
					\item Since TA internalized TOT externality, I think it's possible that ex-ante lobbying undoes some of that--not directly, but raises tariff above the level that is applied when TOT completely internalized. I'm not convinced though.
				\end{itemize}
		\end{itemize}
	\item Bargaining power
		\begin{itemize}
			\item It's important in MRC's story: they show how results change if they change bargaining power, and it's dramatic b/c they take an extreme view that lobbies have all the power
			\item I view bargaining power in this environment as an artificial construct---what's going on isn't really bargaining; bargaining is used as a reduced form. But it's a much more complex process
				\begin{itemize}
					\item But bargaining power can be approximated by who gets what share of the rents
					\item What are the ``rents''?
						\begin{itemize}
							\item How does ``surplus'' change from situation with no lobbying?
							\item In my model, with $e=0$, gov't maximizes social welfare
							\item If $W$ is unweighted and no trade agreement, then there is definitely surplus from lobbying:
									\[
									  W(\ga(e),\tau) + \pi(\tau) > W(\ga(0),\tau = optimal tariff ), \pi(optimal tariff)
									\]
							\item Same with TA with lobbying ex-post: gov't won't choose $\tau>0$ unless it improves political welfare.
							\item It's not clear in ex-ante case; I'd have to do more work. It's not clear whether it's possible for gov't welfare to be strictly reduced by ex-ante lobbying (this should be relative to no lobbying? no trade agreement? Could lobbying incentives be such that lobby puts forth so much effort that when $W$ is concave in $\ga$, gov't falls below original level at $\tau = 0$? or $\tau = $ optimal unilateral tariff (depending on counterfactual)
							\item But in first two, there is some intermediate split of surplus
						\end{itemize}
				\end{itemize}
		\end{itemize}
\end{itemize}


\newpage
December 15, 2014
\begin{itemize}
	\item (Dec 18) I'm now thinking of ex-ante-lobbying changing $e^{EP}$ to $\max \left\{e^{EA}, e^{EP}\right\}$
	\item BS2005 appraoch is that TA is meant to optimally anticipate future pressure
	\item BUT when it is entirely endogenous, can use trade-agreement-as-commitment device to control what $e^{EP}$ will be
		\begin{itemize}
			\item i.e. $e^{EP}$ is a function of $\tau^{TA}$
			\item Government DOESN'T have to be controlled by backward inducting, seeing whatever lobby will choose as optimal for itself
			\item So government can choose $\tau^{TA^*}$ that optimizes $\expect \left[W(\ga) + W^* \right]$
			\item TA is used to arrange for the gov'ts' joint optimal ex-post level of $\ga$
			\item $\tau^{TA^*} = \frac{4\left(\ga(e)-1\right)}{25-4\ga(e)}$
		\end{itemize}
	\item Government loses some measure of control when there is ex-ante lobbying.
		\begin{itemize}
			\item If $e^{EA} < e^{EP}$, ex-ante lobbying has no impact
			\item If $e^{EA} > e^{EP}$, $\tau^{TA} >\tau^{TA^*}$
			\item Govts lose some measure of control: if $e^{EA} = \max \left\{e^{EA}, e^{EP}\right\}$, then it determines $\tau^{TA}$
				\begin{itemize}
					\item Gov't still chooses ``optimal'' TA, but it's constrained optimal. Constrained by current pressure.
				\end{itemize}
		\end{itemize}
	\item BUT ex-ante lobbying is more expensive: have to do it twice to get more protection than in ex-post only case
		\begin{itemize}
			\item This will be too expensive for some, so they won't lobby ex-ante
			\item This is part of what trade agreements do: relative to no agreement at all, raises price of protection. If only ex-post lobbying is rational, government gets control
		\end{itemize}
	\item Have to figure out what correct comparisons are:
		\begin{enumerate}
			\item TA relative to no TA?
				\begin{itemize}
					\item If gov't wants more pressure, it should avoid TA. But then loses TOT internalization
					\item If gov't wants less pressure, TA doubly good
						\begin{itemize}
							\item Presumes gov't is free to choose which sectors are part of TA, but this is part of a more complicated game; should be able to get payments to leave them out, then payments for unconstrained protection. But GATT rules...
						\end{itemize}
				\end{itemize}
			\item TA with ex-ante relative to TA with ex-post only?
			\item Cross industry? It's possible that $W$ has same shape in $\ga$ for all industries ($i$), but that the shape of $\ga$ differs by $i$
				\begin{itemize}
					\item Remember what $\ga$ is: reduced form for how political pressure translates into weight put on profits in policy-making process
				\end{itemize}
		\end{enumerate}
\end{itemize}

\vskip1in
Real question: why does the government set non-zero $\tau^{TA}$?
\begin{itemize}
	\item Because $\ga(e^{TA})$ together with $\tau^{TA}$ provides higher welfare than $\ga(0)$ together with $\tau=0$
	\item With ex-ante lobbying, $e^{EP}$ will still equal $e^{TA}$
		\begin{itemize}
			\item Instead of purely looking ahead and getting to manipulate, gov't welfare fcn gets highjacked
			\item I think I need to show several interesting possibilities and leave it at that. What are they?
		\end{itemize}
\end{itemize}

\vskip1in
Lobby faces different incentives ex-ante and ex-post
\begin{itemize}
	\item Ex-post, pays according to $\tau^N$
	\item Ex-ante, pays according to $\tau^E$
\end{itemize}
Therefore it's possible to have $e^{EP} < e^{L}$ even with $e^{EA}>0$
\begin{itemize}
	\item I believe a necessary condition is that government welfare is concave in $\ga(e)$ and $\ga^* < \ga^L$
	\item Investigate: It seems as if it must be the case that $\tau^{EA} < \tau^{EP}$. Should be easy to show.
		\begin{itemize}
			\item If this is true, even though ex-ante lobbying takes something away, as long as government \textit{wants} to restrain lobbying, trade agreement will even with ex-ante (relative to trade war level)
			\item Need to be up front about which assumptions (especially on $\ga$) this requires
		\end{itemize}
\end{itemize}

\newpage
Exact tariff commitment result
\begin{itemize}
	\item We know $e^{EP} =0$ and $\tau^{EP} = \tau^{TA}$ when no possibility of ex-ante lobbying
	\item What does lobby's problem look like?
		\begin{itemize}
			\item At ex-post stage, $e^{EA}$ sunk. So ex-post problem looks the same.
			\item In principle, at ex-ante stage, $e^{EA}$ will have impact on $e^{EP}$. But not in exact tariff case: $e^{EP} = 0 \ \forall e^{EP}$.
			\item So this collapses to a simple problem of maximizing joint gov't welfare with only $e=e^{EA}$
				\begin{itemize}
					\item Looks like a problem with no ex-post lobbying, but where there is perfect enforcement of the tariff commitment
					\item Essentially, there is no future for the government to control
				\end{itemize}
		\end{itemize}
\end{itemize}

\vskip.2in
Tariff cap result
\begin{itemize}
	\item Here, $e^{EP}$ will also work as in body of paper. The earlier stage is sunk, and applied level will match whatever effort lobby puts forth
		\begin{itemize}
			\item It's just that cap could be higher due to ex-ante lobbying
		\end{itemize}
	\item Since $e$ at ex-ante stage is $\max\left\{e^{EA},e^{EP}\right\}$, government sets cap for anticipated $e^{EP}$ or for $e^{EA}$, whichever is higher.
	\item Lobby has to decide optimal $e^{EA}$
		\begin{itemize}
			\item Use calculus condition
				\[
				  \pi(\tau^E(e^{EA})) - e^{EA} - e^{EP}(e^{EA})
				\]
				When this FOC is operative, government decides tariff cap according to joint decision problem with $e^{EA}$, but then lobby still has to exert effort up to that cap; $e^{EP}$ is the amount the lobby pays according to $\tau^N$ to achieve $\tau^E(e^{EA})$
					\begin{itemize}
						\item FOC
							\[
							  \frac{\partial \pi}{\partial \tau} \frac{\partial \tau}{\partial \ga} \frac{\partial \ga}{\partial e^{EA}} = 1 +\frac{\partial e^{EP}}{\partial e^{EA}}
							\]
					\end{itemize}
			\item Then, set to zero if either $e^{EA} < e^{EP}$ or
			  \[
				  \pi(\tau(e^{EA})) - e^{EA} - e^{EP} < \pi(\tau(e^{EP})) - e^{EP} 
				\]
				where $\tau$ is set according to $\tau^E$ (have to be super careful about how to define these $e$'s).
			\item Intuition: If lobby's incentives to exert pressure ex-ante are strong enough, they overpower the government's ability to use the TA to directly manipulate lobbying incentives ex-post
				\begin{itemize}
					\item Lobby can force government to set cap higher than the level that would optimize its welfare when there is no ex-ante lobbying
					\item BUT the TA itself mutes ex-ante lobbying incentives (through a couple channels, I think)
				\end{itemize}
			\item If government would set $\tau^{TA}$ such that lobby can achieve $e^L$ (that is, this level is optimal for gov't), then there is no incentive to lobby ex-ante
				\begin{itemize}
					\item If gov't sets $\tau^{TA}$ lower than this level, I think it's true that lobby will NOT get $e^L$ due to differing incentives ex-ante (lobbying more expensive.
					\item Could be higher, but not all the way to level that would achieve $e^L$? Or could optimal ex-ante lead to this because ex-post lobbying can achieve more for less?
				\end{itemize}
	   
		\end{itemize}
	 \item It's possible that $e^{EP} < e^{L}$ with $e^{EA} > 0$
		\begin{itemize}
			\item Not sure if $W$ concave in $\ga$ is necessary, but it seems like it
			\item $W$ concave in $\ga$ implies $W$ increases as $\pi_x$ is weighted more heavily (as in standard $W$), but eventually $W$ decreases in $\ga$
				\begin{itemize}
					\item Giving them more weight doesn't give gov't proportionally higher welfare
					\item Something like Bill Ethier's 2012 GJE (Political Economy Approach) paper where there are diminishing returns to lobbying as $\tau$ increases
					\item I can interpret this as $\ga$ being a concave function of $\tau$, but not sure that gets me what I need
					\item diminishing returns to lobbying in $\tau$ for lobbyists vs. diminishing returns to awarding higher $\tau$ for gov't
				\end{itemize}
		\end{itemize}
\end{itemize}

\end{document}
%\documentclass[12pt]{article}

%\addtolength{\textwidth}{1.4in}
%\addtolength{\oddsidemargin}{-.7in} %left margin
%\addtolength{\evensidemargin}{-.7in}
%\setlength{\textheight}{8.5in}
%\setlength{\topmargin}{0.0in}
%\setlength{\headsep}{0.0in}
%\setlength{\headheight}{0.0in}
%\setlength{\footskip}{.5in}
%\renewcommand{\baselinestretch}{1.0}
%\setlength{\parindent}{0pt}
%\linespread{1.1}

%\usepackage{amssymb, amsmath, amsthm, bm}
%\usepackage{graphicx,csquotes,verbatim}
%\usepackage[backend=biber,block=space,style=authoryear]{biblatex}
%\setlength{\bibitemsep}{\baselineskip}
%\usepackage[american]{babel}
%dell laptop
%\addbibresource{C:/Users/Kristy/Dropbox/Research/xBibs/tradeagreements.bib}
%\addbibresource{C:/Users/Kristy/Documents/Dropbox/Research/xBibs/tradeagreements.bib}
%\renewcommand{\newunitpunct}{,}
%\renewbibmacro{in:}{}


%\DeclareMathOperator*{\argmax}{arg\,max}
%\usepackage{xcolor}
%\hbadness=10000

%\newcommand{\ve}{\varepsilon}
%\newcommand{\ov}{\overline}
%\newcommand{\un}{\underline}
%\newcommand{\al}{\alpha}
%\newcommand{\expect}{\mathbb{E}}
%\newcommand{\ga}{\gamma}
%\newcommand{\Ga}{\Gamma}
%\newcommand{\de}{\delta}

%\begin{document}

In many situations, it is realistic to assume that, in addition to the ex-post lobbying examined throughout this text, lobbying also takes place before the trade agreement is signed to affect the terms of the trade agreement. To reiterate what has been shown in previous sections, although ex-post lobbying occurs after the trade agreement is signed in order to impact the level of the applied tariff, it alters the incentives of governments when they negotiate the trade agreement. Similarly, ex-ante lobbying is aimed directly at changing the level of the trade agreement tariff, but here lobbies only engage in this activity in order to alter what will be possible ex-post.

With the addition of the possibility of ex-ante lobbying, the lobby may have to pay twice. As in Section~\ref{sec:strong}, when facing a trade agreement in which commitments take the form of strong bindings, the lobby's optimal ex-post effort choice is zero. However, when facing a tariff cap, protection will only be supplied at the ex-post stage if the lobby exerts effort. Thus, ex-ante lobbying adds a second time at which the lobby may have an incentive to exert effort in the case of a tariff cap; it introduces a first time at which the lobby may supply non-zero effort in the case of a strong binding.

For simplicity, here I consider ex-ante lobbying when political pressure is endogenous only, i.e. there is no exogenous, probabilistic influence on $\ga$. Since there is no possibility of ex-ante lobbying in the \Textcite{bs2005} framework this paper follows, it must be determined how ex-ante lobbying can best be integrated into the model.

\Textcite{bs2005} posits that governments negotiate the trade agreement so that they will be able to set the applied tariff that best matches the political pressure they experience ex-post. In the case of endogenous political pressure, I have argued that the governments will tune the trade agreement to deliver the amount of lobbying effort that is in their best interest ex-post. How does this change when lobbies can influence the terms of the trade agreement?

In the case of lobbies, a lobby knows that the trade agreement will bind its future action, so exerting effort at the ex-ante stage only matters insofar as it improves ex-post profits net of both ex-ante and ex-post lobbying effort. For the governments, with ex-post pressure only, the governments are completely forward looking: they set the trade agreement with a view toward maximizing expected joint ex-post welfare. In order for ex-ante lobbying to have an effect, it must be that governments also receive some utility from the setting of the trade agreement itself: perhaps they receive or lose political capital from various interested parties, and can be persuaded to forego future benefits if ex-ante lobbying is strong enough to outweigh those future concerns. 

Thus I model the influence of ex-ante lobbying on the governments' tariff-setting behavior as operating through a threshold. Recall how the trade agreement tariff is determined under ex-post lobbying only. Because the trade agreement acts as a commitment device, the governments choose the tariff level $\tau^R_{W,e}$, and implicitly $e^{EP}=e^R_{W,e}$, to maximize their joint political welfare as in Equation~\ref{exp:1}, subject only to the constraint that $\ga$ will be realized according to the ex-post lobbying process.
		
As discussed in Section~\ref{sec:objfcn}, depending on how government welfare changes as political pressure increases, the governments may be looking ahead and manipulating ex-post lobbying incentives when there is only ex-post lobbying. When ex-ante lobbying is possible the governments may have to raise the trade agreement tariff above $\tau^R_{W,e}$ in order to respond to that pressure.

This is modeled here with a simple cut-off. The trade agreement tariff cap will be set according to $\ga\left(e^{EA}\right)$ instead of at $\tau^R_{W,e}$ if $\ga\left(e^{EA}\right)$ would lead to a higher trade agreement tariff. If, on the other hand, $\tau = \frac{4(\ga(e^{EA})-1)}{25-4\ga(e^{EA})} < \tau^R_{W,e}$, the governments respond to the higher ex-post pressure and set $\tau^R_{W,e}$.

Thus the governments lose some measure of control with ex-ante lobbying. If $e^{EA}$ is low enough, ex-ante lobbying has no impact. But if $e^{EA}$ is above the threshold, the trade agreement tariff is strictly greater than $\tau^R_{W,e}$ and so joint government welfare will not be at its globally optimal level. The trade agreement the governments choose will be constrained by the ex-ante pressure.

We will see that this change to the government's preferences has the impacts one would expect, at least in the case of tariff caps: net profits for the lobby (weakly) increase, and net political utility for the governments (weakly) decrease.

\subsection{Strong Bindings}
Let us first look at how ex-ante lobbying impacts trade agreements that involve strong bindings, or exact tariff commitments. We know that when there is only ex-post lobbying, $e^{EP} =0$ and $\tau^{EP} = \tau^R_{S,e}$. Since ex-ante effort will be sunk ex-post, the ex-post stage of the game that includes ex-ante lobbying is identical. In the case of exact tariff commitments, because $e^{EP} = 0 \ \forall e^{EA}$, the lobby does not consider the impact of ex-ante lobbying on the ex-post effort level.

This thus looks like a problem with no ex-post lobbying, but where there is perfect enforcement of the tariff commitment. This collapses to a simple problem of maximizing joint government welfare with $e=e^{EA}$, and the lobby maximizing profits net of ex-ante effort facing the joint maximization of the governments.

Without knowing how government welfare varies in $\ga$, we can't say for sure the effect on government welfare at the ex-ante stage. As with the example in Figure~\ref{fig:weight}, joint welfare could fall but could also rise. Ex-post political pressure will not match ex-ante political pressure: because of the incentives involved with exact tariff commitments, the home government will experience $\ga(0)$ but be forced to apply the ex-ante tariff, so this is a welfare loss relative to the no ex-ante case at this stage.\footnote{This is a government welfare loss from exact tariff commitments that is not accounted for in \Textcite{mrc2007} because of their assumption that the government has no bargaining power. It seems to be closely related that, in their model,  exact tariff commitments and tariff caps are equivalent in the case of perfectly immobile capital. As will be shown below, that is not the case here.} There is a very clear gain for the lobby; it there were not, it can always set $e^{EA}=0$.
				
\subsection{Weak Bindings}
\begin{itemize}
	\item Here, $e^{EP}$ will also work as in body of paper. The earlier stage is sunk, and applied level will match whatever effort lobby puts forth
		\begin{itemize}
			\item It's just that cap could be higher due to ex-ante lobbying
				\begin{itemize}
					\item Without ex-ante lobbying, government chooses $\tau^R_{W,e}$ that leads to ex-post $e$ that maximizes ex-ante joint welfare
		\[
		  W(\ga(e),\tau^R_{W,e}) + W^*(\tau^R_{W,e})
		\]
		knowing they'll set $\tau^N$ in the third stage (i.e. $\tau^N(\ga(e^R_{W,e})) = \frac{8\ga(e^R_{W,e}) - 5}{68 - 8 \ga(e^R_{W,e})} = \tau^R_{W,e}$).
						\begin{itemize}
							\item $\tau^R_{W,e}$ is the optimal unilateral choice of the home government when the lobby is faced with the constraint of the trade agreement
							\item Lobby would max $\pi(\tau(\ga(e))) -e$; instead sets $e = e^R_{W,e} \Rightarrow \tau^N(\ga(e^R_{W,e})) =\tau^R_{W,e}$
						\end{itemize}
					\item Government actually ends up with a lower $e$ mapping into a lower $\ga$ to produce the trade agreement $\tau^R_{W,e}$ according to $\tau^N$, so has to set $\tau^R_{W,e}$ higher to target preferred $\ga$ according to joint max problem [in this symmetric case, identical to unilateral problem]
			\end{itemize}
		\end{itemize}
	\item Lobby decides $e^{EA}$ given that gov't will set cap $\tau^R_{W,e}$ or in ex-ante stage given $e^{EA}$, whichever is higher
		\begin{itemize}
			\item Start with calculus condition
				\[
				  \pi(\tau^E(e^{EA})) - e^{EA} - e^{EP}(e^{EA})
				\]
				When this FOC is operative, government decides tariff cap according to joint decision problem with $e^{EA}$, but then lobby still has to exert effort up to that cap; $e^{EP}$ is the amount the lobby pays according to $\tau^N$ to achieve $\tau^E(e^{EA})$
					\begin{itemize}
						\item FOC
							\begin{equation}
							  \frac{\partial \pi}{\partial \tau} \frac{\partial \tau^E}{\partial \ga} \frac{\partial \ga}{\partial e^{EA}} = 1 +\frac{\partial e^{EP}}{\partial e^{EA}}
								\label{eq:ea}
							\end{equation}
					\end{itemize}
			\item Then, set to zero if
			  \begin{equation}
				  \pi(\tau^E(e^{EA})) - e^{EA} - e^{EP}(e^{EA}) < \pi(\tau^N(e^R_{W,e})) - e^R_{W,e} 
					\label{ine:ea}
				\end{equation}
			\item Intuition: If lobby's incentives to exert pressure ex-ante are strong enough, they overpower the government's ability to use the trade agreementto directly manipulate lobbying incentives ex-post
				\begin{itemize}
					\item Lobby can force government to set cap higher than the level that would optimize its welfare when there is no ex-ante lobbying
					\item Result: With tariff cap, ex-ante lobbying can only (weakly) reduce gov't political welfare. It can only (weakly) increase net profits for lobby (lobby can set $e^{EA}=0$ if ex-ante lobbying would make it worse off)
					\item i.e. tariff with ex-ante lobbying can only be (weakly) higher than without ex-ante lobbying
				\end{itemize}	
		\end{itemize}	
\end{itemize}	
				
%\vskip.5in
\newpage
\begin{itemize}					
	\item $\tau^{EA} < \tau^N(e^N)$
		\begin{itemize}
			\item Lemma to show ex-post lobbying leads to lower tariff than no agreement
			\item Denote tariff with no agreement as $\tau^N(e^N)$. Then the trade agreement tariff with ex-ante lobbying, $\tau^{EA} \geq \tau^N(e^N)$. (this will be lemma) \\
			Proof: The lobby's effort level in the absence of a trade agreement is determined by the first order condition
				\begin{equation}
						\frac{\partial \pi}{\partial \tau} \frac{\partial \tau^N}{\partial \ga} \frac{\partial \ga}{\partial e^N} = 1
						\label{eq:nta}
				\end{equation}
			In order to compare this with Equation (\ref{eq:ea}), note that
			  \[
				  \frac{\partial e^{EP}}{\partial e^{EA}} = \frac{\partial e^{EP}}{\partial \tau^{EA}} \frac{\partial \tau^{EA}}{\partial e^{EA}} = \frac{1}{\frac{\partial \tau^N}{\partial e^{EP}}} \frac{\partial \tau^{E}}{\partial e^{EA}} =  \frac{\frac{\partial \tau^{E}}{\partial e^{EA}}}{\frac{\partial \tau^N}{\partial e^{EP}}} =  \frac{\frac{\partial \tau^{E}}{\partial \ga}\frac{\partial \ga}{\partial e}}{\frac{\partial \tau^N}{\partial \ga}\frac{\partial \ga}{\partial e}} =  \frac{\frac{\partial \tau^{E}}{\partial \ga}}{\frac{\partial \tau^N}{\partial \ga}} 
				\]
				where the first equality follows by the fact that the impact of a marginal unit of ex-ante lobbying effort on ex-post effort is to raise the trade agreement tariff. This is a tariff cap, and so we need to calculate the effect of this increased cap on ex-post effort. The second equality represents that this is just the inverse of the marginal impact of ex-post lobbying effort on the non-cooperative tariff; the super-script on the $\tau$ in the denominator has been changed to $N$ to denote that ex-post tariff setting is non-cooperative; that on the $\tau$ in the numerator has been change to $E$ to denote that trade agreement tariff setting is cooperative. The fourth equality unpacks the partial derivatives to show the intermediate impact of $\ga$. This leads to the final expression in which the common terms have been canceled. \\
				Multiplying Equation (\ref{eq:ea}) by $\frac{\frac{\partial \tau^N}{\partial \ga}}{\frac{\partial \tau^E}{\partial \ga}}$
				  \begin{equation}
					  \frac{\partial \pi}{\partial \tau} \frac{\partial \tau^N}{\partial \ga} \frac{\partial \ga}{\partial e^{EA}} = \frac{\frac{\partial \tau^N}{\partial \ga}}{\frac{\partial \tau^E}{\partial \ga}} + 1
						\label{eq:ea2}
					\end{equation}
				Because $\frac{\frac{\partial \tau^N}{\partial \ga}}{\frac{\partial \tau^E}{\partial \ga}} > 0$, the solution to the ex-ante first order condition as given in Equation~(\ref{eq:ea2}) is smaller than the solution to Equation~(\ref{eq:nta}), so that the lobby exerts less effort in the ex-ante stage of trade agreement negotiations than when there is no trade agreement. Since $\tau^E(\cdot) < \tau^N(\cdot)$ for all $e$, $\tau^{EA} < \tau^N(e^N)$. \\
			(Important to give intuition since increased incentive to exert lobbying effort...)
				\item If government would set $\tau^R_{W,e}$ such that lobby can achieve $e^L$ (that is, this level is optimal for lobby), then there is no incentive to lobby ex-ante
		\begin{itemize}
			\item If gov't sets $\tau^R_{W,e} < e^L$, lobby cannot get $e^L$ due to differing incentives ex-ante 
						\begin{itemize}
							\item Ex-post, pays according to $\tau^N$, ex-ante, pays according to $\tau^E$
						\end{itemize}
					\item Could be higher, but not all the way to level that would achieve $e^L$
				\end{itemize}
			\item Result: then tariffs are always lower with a trade agreement.
			Proof: It is shown above that tariffs under ex-ante lobbying are set either in line with lobbying effort according to the first order condition (\ref{eq:ea}) or via ex-post incentives. 
				\begin{itemize}
					\item Assume throughout that $\frac{\partial \ga}{\partial e} > 0$
				\end{itemize}
			\item Even though ex-ante lobbying takes something away, as long as government \textit{wants} to restrain lobbying, trade agreement will do this even with ex-ante lobbying (relative to trade war level)
				\begin{itemize}
					\item If gov't wants more pressure, it should avoid trade agreement. But then loses TOT internalization
					\item If gov't wants less pressure, trade agreementdoubly good
						\begin{itemize}
							\item Presumes gov't is free to choose which sectors are part of trade agreement, but this is part of a more complicated game; should be able to get payments to leave them out, then payments for unconstrained protection. But GATT rules...
							\item Cross industry, it's possible that $\ga$ affects $W$ the same for all industries ($i$), but that the shape of $\ga$ differs by $i$
							\item Remember what $\ga$ is: reduced form for how political pressure translates into weight put on profits in policy-making process
							\item This would inform who gets included / left out of trade agreements
						\end{itemize}
					\end{itemize}
		\end{itemize}
	\item trade agreementmutes lobbying incentives, even if there is ex-ante lobbying
		\begin{itemize}
			\item ex-ante lobbying is more expensive: both on the margin, and because you have to pay twice in order to get more protection than in ex-post only case
			\item This will be too expensive for some, so they won't lobby ex-ante (see Condition~\ref{ine:ea})
			\item This is part of what trade agreements do: relative to no agreement at all, raises price of protection. If only ex-post lobbying is rational, government gets control
		\end{itemize}

		
	 \item It's possible that $e^{EP} < e^{L}$ with $e^{EA} > 0$
		\begin{itemize}
			\item For instance, take $W$ concave in $\ga$ (and $\ga^* < \ga^L$)
			\item This implies $W$ increases as $\pi_x$ is weighted more heavily (as in standard $W$), but eventually $W$ decreases in $\ga$
				\begin{itemize}
					\item Giving them more weight doesn't give gov't proportionally higher welfare
					\item Something like Bill Ethier's 2012 GJE (Political Economy Approach) paper where there are diminishing returns to lobbying as $\tau$ increases
					\item I can interpret this as $\ga$ being a decreasing function of $\tau$
					%\item diminishing returns to lobbying in $\tau$ for lobbyists vs. diminishing returns to awarding higher $\tau$ for gov't
				\end{itemize}
		\end{itemize}
\end{itemize}



		
\newpage
\vskip.2in
Compare to ex-ante lobbying \`{a} la MRC.
\begin{itemize}
	\item Cooperative bargaining soln (maximize joint welfare of both governments and both lobbies) instead of extensive form
		\begin{itemize}
			\item leads to bizarre outcome where government will take into account lobby's profits even if $e^{\text{EA}}=0$.
		\end{itemize}
	\item In MRC main treatment, assume lobby has all the bargaining power
		\begin{itemize}
			\item Implies tariff caps and strong bindings equivalent, even though these split the surplus in very different ways. All surplus goes to lobby
			\item In the part that is only in the NBER working paper, they have
				\begin{itemize}
					\item $z=0$ (perfectly immobile capital)
					\item no ex-ante lobbying
					\item government has all the bargaining power
					\item government sometimes chooses free trade---so there IS an incentive to use the trade agreement as a domestic commitment device
				\end{itemize}
			\end{itemize}
	\item In MRC, no matter how tariff is set (cooperatively or non-cooperatively), contribution is $xp(\ov{t})$
		\begin{itemize}
			\item Not true in GH95
			\item In my set-up, lobby's incentives change when tariff-setting process changes
		\end{itemize}
	\item In MRC, weight on lobby's profits is $1+a$ no matter what. Doesn't vary in lobby's efforts. GH not really microfoundations for flexible model as in Long and Vousden (1991) / Baldwin (1987) model; microfoundations for more restrictive version with fixed weights; then decide $\tau$ to change PS, CS, etc. (p. 481 GH94). Can't get shocks as in this literature
		\begin{itemize}
			\item I still don't know exactly why you never want to reign in lobbying in GH; because it's just a conduit for information, and gov't is fully compensated for loss? But is it?
		\end{itemize}
	\item Bargaining power
		\begin{itemize}
			\item It's important in MRC's story: they show how results change if they change bargaining power, and it's dramatic b/c they take an extreme view that lobbies have all the power
			\item I view bargaining power in this environment as an artificial construct---what's going on isn't really bargaining; bargaining is used as a reduced form. But it's a much more complex process
		\end{itemize}
\end{itemize}

		
%\end{document}
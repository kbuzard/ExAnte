Much of the work on the political economy of trade agreements focuses on questions of the optimal design of trade agreements, trade agreement negotiations, and trade dispute settlement that arise in the presence of asymmetric information about shocks to an exogenous political economy parameter. 

One of the basic ideas that emerges from this literature is that, in the presence of asymmetric information about the strength of the ex-post political economy shocks, it is often advantageous to grant governments a period of relief from trade commitments. That is, one would rather allow a short period of ``escape'' from the agreement rather than have the agreement abandoned forever because domestic political opposition is temporarily too strong to be resisted.

Various trade agreement and institutional design features have been proposed in the literature that can help to make an escape clause work, since it is not incentive compatible to allow governments to take advantage of an escape clause whenever they wish. If some cost can be associated with the use of the escape clause or a dispute settlement body (DSB) can procure a signal about the strength of $\ga$, it is often advantageous to grant this period of relief from trade commitments if the signal is strong enough. 

%In this vein, \Textcite{bs2005} find that governments prefer tariff caps to strong bindings when there is no escape clause, but that impatient governments may need an escape clause in order to make cooperation sustainable. 

This is an intuitively appealing story, but the logic can break down in the presence of endogenous political pressure. An escape clause allows a government to apply a higher tariff barrier when it experiences intense political pressure. But if a government gets a free pass from the WTO whenever it feels sufficient political pressure from domestic interest groups, those interest groups have a strong incentive to exert the required level of pressure. I show that, depending on the government's objectives, this extra political pressure may reduce the government's political welfare.

I employ a model that closely follows \Textcite{bs2005}, adding an endogenously-determined element to their exogenously-determined political economy weights. I show that this can explain why the conditions for invoking the WTO Safeguards measure are purely observable economic variables, and why the level of protection governments can choose when invoking a Safeguard is not restricted.

I explore the impacts the elevated levels of political pressure that an escape clause might encourage on governments' political welfare. To do this, I compare welfare under the standard Baldwin-style government welfare function and a similar, weighted version. The slight modification to the government objective function demonstrates that governments may want to use trade agreements to reduce lobbying. Thus, examining this alternative welfare function in combination with endogenous lobbying can provide a bridge between the theoretical literature and the claims of trade policy practitioners that an important role of trade agreements is to rein in protectionist pressure.

This generalizes the result from \Textcite{mrc2007} that trade agreements can be used as a domestic political commitment device. In their setting, the political commitment role of trade agreements is present only when capital has some degree of mobility;  here this restriction is not needed to establish the result. In a further generalization of \Textcite{mrc2007}, I show that one of the uses of tariff caps is to incentivize the lobby to engage in the political process after the trade agreement is in place. The lobby loses all incentive to exert effort under a strong binding and so a government who prefers some level of political engagement from the lobby will not use a strong binding. Again, this does not depend on a particular capital mobility structure, so this, perhaps, provides an explanation for the ubiquity of tariff caps. Combined with the idea that governments can use tariff caps to restrain endogenous political pressure, a story emerges in which governments employ trade agreements to carefully manipulate lobbying incentives in order to maximize their political objectives.

A rich literature has been developed to address questions concerning the design and enforcement of trade agreements. Repeated non-cooperative game models of trade agreements have been considered by \Textcite{mcm86,mcm89}, \Textcite{dixit1987}, \Textcite{bs1990, bs1997a, bs1997b, bs2002}, \Textcite{kovthurs}, \Textcite{maggi99}, \Textcite{ederington}, \Textcite{rosendorff}, \Textcite{bagwell2009}, and \Textcite{park}.

Maggi and Staiger have a series of papers that employ an exogenous political economy force to study questions about the design of trade agreements and trade dispute settlement. \Textcite{ms2012a} study the conditions under which property versus liability rules will be optimal when renegotiation of agreements is possible. They find that when property rules are optimal, agreements are never renegotiated; only liability rules are renegotiated in equilibrium, and when this renegotiation occurs, it always results in trade liberalization. \Textcite{ms2013} builds on this work to answer questions about when governments will settle disputes and how this relates to the contracting environment. \Textcite{ms2011} has a more sophisticated set-up for the exogenous shock that allows the authors to speak to issues of the role of the dispute settlement body as interpreter and completer of incomplete contracts.
		
\Textcite{beshkar2010a} shows that when one assumes that utility is not transferable between countries as has become common in the literature, the optimal mechanism involves less-than-proportional retaliation against parties who have defected from the agreement. \Textcite{beshkar2010b} compares the GATT escape clause to the WTO Safeguards agreement and shows that the DSB as a non-binding arbitrator can assist governments in self-enforcing their trade agreements. \Textcite{martinvergote} demonstrate that future punishment provides for higher welfare than contemporaneous punishment when governments are sufficiently patient. Indeed, they show that retaliation is a necessary feature of any efficient equilibrium in this environment. \Textcite{hungerford} and \Textcite{riezman1991} also consider the impact of different assumptions about reactions and timing of punishments for deviations from agreements.

This work is also related to the literature on the endogenous political economy of trade. The foundational work is \Textcite{gh94}; the insights are applied to trade agreements in \Textcite{gh95}. \Textcite{mrc2007} advanced the literature by demonstrating that there is a domestic commitment role for trade agreements. \Textcite{buzard2013a} features a repeated-game model similar in spirit to the model under consideration here but focusing on questions of optimal punishments when the government has a separation-of-powers structure, while \Textcite{coatesludema} demonstrate that, in the presence of opposition to a trade agreement from foreign lobbies, it may be optimal to liberalize trade unilaterally.

In the next section, I present the model and some preliminary results. Section~\ref{sec:rigid} contains the analysis of rigid tariffs while Section~\ref{sec:escape} explores trade agreements with escape clauses. The political welfare function of the government is examined in depth in Section~\ref{sec:objfcn} before returning to examine the escape clause in an environment with both endogenous and exogenous political pressure in Section~\ref{sec:escape2}. Section~\ref{sec:concl} concludes.
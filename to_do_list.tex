\documentclass[12pt]{article}

\addtolength{\textwidth}{1.4in}
\addtolength{\oddsidemargin}{-.7in} %left margin
\addtolength{\evensidemargin}{-.7in}
\setlength{\textheight}{8.5in}
\setlength{\topmargin}{0.0in}
\setlength{\headsep}{0.0in}
\setlength{\headheight}{0.0in}
\setlength{\footskip}{.5in}
\renewcommand{\baselinestretch}{1.0}
\setlength{\parindent}{0pt}
\linespread{1.1}

\usepackage{amssymb, amsmath, amsthm, bm}
\usepackage{graphicx,csquotes,verbatim}
\usepackage[backend=biber,block=space,style=authoryear]{biblatex}
\setlength{\bibitemsep}{\baselineskip}
\usepackage[american]{babel}
%dell laptop
\addbibresource{C:/Users/Kristy/Dropbox/Research/xBibs/tradeagreements.bib}
%\addbibresource{C:/Users/Kristy/Documents/Dropbox/Research/xBibs/tradeagreements.bib}
\renewcommand{\newunitpunct}{,}
\renewbibmacro{in:}{}


\DeclareMathOperator*{\argmax}{arg\,max}
\usepackage{xcolor}
\hbadness=10000

\newcommand{\ve}{\varepsilon}
\newcommand{\ov}{\overline}
\newcommand{\un}{\underline}
\newcommand{\ta}{\theta}
\newcommand{\al}{\alpha}
\newcommand{\Ta}{\Theta}
\newcommand{\expect}{\mathbb{E}}
\newcommand{\Bt}{B(\bm{\tau^a})}
\newcommand{\bta}{\bm{\tau^a}}
\newcommand{\btn}{\bm{\tau^n}}
\newcommand{\ga}{\gamma}
\newcommand{\Ga}{\Gamma}
\newcommand{\de}{\delta}

\begin{document}
\begin{center}
Ex-ante Lobbying
\end{center}
%\chead{to_do_list.tex}

\vskip.2in
\begin{itemize}
	\item intro
		\begin{enumerate}
			\item Need to take endogenous effort into account for some questions
			\item Role of trade agreements
				\begin{enumerate}
					\item Baldwin-style gov't welfare function works for a set of questions, but not for this one
						\begin{itemize}
							\item Need to be able to say which predictions/design questions are okay: those where it doesn't matter whether government cares if it encourages/discourages lobbying: e.g. already described (ms2011), ms2012a, 
						\end{itemize}
					\item In MRC, weight on lobby's profits is $1+a$ no matter what. Doesn't vary in lobby's efforts. GH not really microfoundations for flexible model as in Long and Vousden (1991) / Baldwin (1987) model; microfoundations for more restrictive version with fixed weights ($1+a$ on those who lobby, $a$ on those who don't); then decide $\tau$ to change PS, CS, etc. (p. 481 GH94). Can't get shocks as in this literature
					\item MRC-style results survive incorporation of ex-ante lobbying. TA still works to restrict lobbying in many cases, but also to keep lobbyists ``in the game''
					\item Nuanced view of domestic commitment motive: it's there in ex-post, but they lose control in ex-ante. Still, with ex-ante TA helps to screen out some lobbies for whom ex-ante is too expensive
				\end{enumerate}
			\item Get implications for real-life design of escape clause
				\begin{enumerate}
					\item What if the WTO actually \textit{did} what the literature says it does? It wouldn't work
				\end{enumerate}
		\end{enumerate}
	\item Conclusion
		\begin{itemize}
			\item perhaps re-emphasize that gov't can't completely control lobbying: can't make it higher than lobby's optimal. But this is b/c there's only one lobby
			\item Be more explicit about relationship between $\ga(e)$ and $\ga(s)$
				\begin{enumerate}
					\item Need to convey the possibilities of this set-up for capturing the real life dynamics of shocks integrated with lobbying dynamics, how that interacts with enforcement and administered protection
				\end{enumerate}
		\end{itemize}
	\item finish escape clause result
\end{itemize}

\vskip.5in
For after this draft
\begin{itemize}
	\item Could escape clause be made to work with some kind of dynamic use constraint?
		\begin{itemize}
			\item When would lobby exert effort to top up?
		\end{itemize}
	\item What units are $\pi(\tau)$ and $e$ measured in? (no numeraire)
	\item existence proofs
\end{itemize}

\end{document}